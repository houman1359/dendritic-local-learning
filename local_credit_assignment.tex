\PassOptionsToPackage{numbers,sort&compress}{natbib}
\documentclass{article}
\usepackage{neurips_2025}
\usepackage{microtype}
\usepackage{amsmath}
\usepackage{amssymb}
\usepackage{bm}
\usepackage{amsthm}
\usepackage{algorithm}
\usepackage{algorithmic}
\usepackage{booktabs}
\usepackage{graphicx}
\usepackage{subcaption}
\usepackage{xcolor}
\usepackage[hidelinks,hypertexnames=false,bookmarks=false]{hyperref}
\graphicspath{{figures/}{./figures/}}

\newtheorem{theorem}{Theorem}
\newtheorem{proposition}{Proposition}
\newtheorem{lemma}{Lemma}
\newtheorem{definition}{Definition}
\newtheorem{remark}{Remark}
\newtheorem{corollary}{Corollary}

\DeclareMathOperator*{\argmax}{arg\,max}
\DeclareMathOperator*{\argmin}{arg\,min}
\DeclareMathOperator{\Var}{Var}
\DeclareMathOperator{\Cov}{Cov}
\DeclareMathOperator{\tr}{tr}

\title{Local Credit Assignment in Compartmental Dendritic Networks}
\author{
Anonymous Authors \\
Paper under double-blind review
}
\date{}

\begin{document}
\maketitle

\begin{abstract}
Backpropagation is effective but biologically implausible due to its reliance on global error transport and weight symmetry. We show that compartmental dendritic networks with shunting inhibition create a regime where strictly local learning rules can approach backpropagation-quality credit assignment. Starting from conductance-based dendritic equations, we derive exact backpropagation gradients for arbitrary dendritic tree morphologies and construct a hierarchy of local approximations---3-factor (3F), 4-factor (4F), and 5-factor (5F) rules---that use only synapse-local quantities plus a broadcast error signal. Our key finding is that \emph{shunting} (conductance-based) inhibition is critical: it provides divisive normalization that dramatically improves the alignment between local and exact gradients. A component-wise gradient-fidelity diagnostic shows that shunting networks achieve $30\times$ higher directional alignment and $10\times$ lower scale mismatch than additive controls. The best local rule (5F with per-soma broadcast) reaches $0.914$ test accuracy on MNIST, closing much of the gap to the $0.965$ backpropagation ceiling on the same architecture.
\end{abstract}

\section{Introduction}

Credit assignment in deep networks is accurate with backpropagation but biologically implausible: it requires global error transport through the exact transpose of forward weights, and symmetric forward-backward pathways that have no known biological substrate. Dendritic neurons suggest an alternative architecture for learning. Real neurons possess spatially extended dendritic trees where each synapse has access to rich local state---driving forces, conductances, and branch-specific context---while global supervision may be reduced to a low-bandwidth modulatory broadcast from the soma.

This paper asks a concrete question: \emph{can dendritic structure and shunting inhibition create regimes where strictly local learning rules approach the credit assignment quality of backpropagation?} We answer affirmatively. Starting from conductance-based dendritic equations, we derive exact backpropagation gradients for dendritic trees and construct a hierarchy of biologically-local approximations (3-factor, 4-factor, and 5-factor rules). We then show empirically that shunting inhibition---which provides divisive normalization of compartment voltages---is the key architectural feature enabling effective local credit assignment. A novel gradient-fidelity diagnostic reveals that shunting networks produce local gradients with $30\times$ better directional alignment and $10\times$ lower scale mismatch compared to additive controls.

\paragraph{Contributions.}
\begin{enumerate}
\item \textbf{Exact dendritic backpropagation and local approximations.} We derive the exact loss gradient for arbitrary dendritic tree morphologies and construct a principled hierarchy of local approximations (3F/4F/5F) that use only quantities available at each synapse.
\item \textbf{Shunting enables local credit assignment.} We show that shunting (conductance-based) inhibition creates a regime where local rules produce gradients that are substantially better aligned with backpropagation than in additive networks, and that this translates to large performance gains.
\item \textbf{Gradient-fidelity diagnostic.} We introduce a component-wise local-vs-backprop gradient comparison that decomposes alignment by parameter type (excitatory, inhibitory, dendritic), providing mechanistic insight beyond end-task accuracy.
\end{enumerate}

\begin{figure*}[t]
\centering
\includegraphics[width=0.95\textwidth]{fig_model_schematic.pdf}
\caption{\textbf{Model overview.} (A) A compartmental dendritic neuron where \emph{each branch} receives both excitatory ($E_j > 0$, blue) and inhibitory ($E_j = 0$, red) synaptic inputs via separate sparse connectivity (TopK). Inhibitory conductances enter only the denominator of the voltage equation (shunting/divisive normalization). Dendritic branch voltages propagate toward the soma via learned dendritic conductances (green). The steady-state voltage at each compartment is a conductance-weighted average (Eq.~\ref{eq:voltage}). (B) Local learning rules of increasing complexity: 3-factor (pre-synaptic activity $\times$ driving force $\times$ broadcast error), 4-factor (+ variance modulator $\rho$), and 5-factor (+ information-theoretic factor $\phi$). The same rule applies to both E and I synapses; the sign difference arises from the driving force $(E_j - V_n)$. The broadcast error $\delta$ can operate in scalar, per-soma, or local mismatch mode.}
\label{fig:model_schematic}
\end{figure*}

\section{Compartmental Voltage Model}

\subsection{From the Passive Cable to the Compartment Equation}
\label{sec:cable-derivation}

Starting from the linear passive cable equation for membrane potential $v(x,t)$ relative to rest,
\[
c_m \frac{\partial v}{\partial t} = \frac{1}{r_a}\frac{\partial^2 v}{\partial x^2} - \frac{1}{r_m} v + i_\mathrm{syn}(x,t),
\]
and discretizing a dendritic branch into isopotential compartments with axial conductances, the steady-state ($\partial_t v=0$) yields a nodal balance of conductances and driving forces.\footnote{See Koch's \emph{Biophysics of Computation} and Dayan \& Abbott's \emph{Theoretical Neuroscience} for derivations and assumptions underlying the linear regime.}
Representing synapses as conductances with reversal potentials and siblings as dendritic conductances gives the compartment equation below with unit leak to $E_\mathrm{leak}{=}0$. This clarifies that (i) all inputs contribute via conductances, (ii) total conductance $g_n^{\rm tot}$ controls both input resistance $R_n^{\rm tot}$ and divisive normalization, and (iii) shunting inhibition corresponds to adding conductance with $E_\mathrm{inh} \approx 0$ (Section~\ref{sec:shunting}).

\subsection{Voltage Equation}

Consider compartment $n$ receiving synaptic inputs indexed by $j$ and dendritic inputs from child compartments. Let:
\begin{itemize}
\item $x_j \in \mathbb{R}_+$: presynaptic activity at synapse $j$
\item $E_j \in \mathbb{R}$: reversal potential of synapse $j$ (excitatory: $E_j > 0$; inhibitory: $E_j \leq 0$)
\item $g_j^{\mathrm{syn}} \geq 0$: synaptic conductance (learned parameter)
\item $V_j \in \mathbb{R}$: voltage of child compartment $j$
\item $g_j^{\mathrm{den}} \geq 0$: dendritic conductance from child $j$ (learned parameter)
\end{itemize}

\paragraph{Currents.}
Synaptic current:
\begin{equation}
I_{\mathrm{syn}} = \sum_j (E_j - V_n) x_j g_j^{\mathrm{syn}}
\end{equation}
Dendritic current:
\begin{equation}
I_{\mathrm{den}} = \sum_j (V_j - V_n) g_j^{\mathrm{den}}
\end{equation}

\paragraph{Steady-state voltage.}
With unit leak conductance to reversal potential $0$:
\begin{equation}
V_n = \frac{\sum_j E_j x_j g_j^{\mathrm{syn}} + \sum_j V_j g_j^{\mathrm{den}}}{\sum_j x_j g_j^{\mathrm{syn}} + \sum_j g_j^{\mathrm{den}} + 1}
\label{eq:voltage}
\end{equation}

\paragraph{Total conductance and resistance.}
\begin{equation}
g_n^{\mathrm{tot}} = \sum_j x_j g_j^{\mathrm{syn}} + \sum_j g_j^{\mathrm{den}} + 1, \qquad R_n^{\mathrm{tot}} = \frac{1}{g_n^{\mathrm{tot}}}
\label{eq:conductance}
\end{equation}

\begin{lemma}[Convexity and Bounds]
\label{lem:convex}
Let $\mathcal{S}_n=\{E_j\}_{\text{syn at }n}\cup \{V_j\}_{\text{children}} \cup \{0\}$. Then $V_n$ in \eqref{eq:voltage} is a convex combination of elements of $\mathcal{S}_n$, hence
\[
\min \mathcal{S}_n \ \le\ V_n \ \le\ \max \mathcal{S}_n.
\]
Moreover, $0<R_n^{\mathrm{tot}}\le 1$ and $R_n^{\mathrm{tot}} g_{i}^{\mathrm{den}}<1$ for all $i$.
\end{lemma}
\begin{proof}
Immediate from \eqref{eq:voltage}–\eqref{eq:conductance} since all conductances are nonnegative and leak adds $+1$ to the denominator.
\end{proof}

\subsection{Local Sensitivities}

\begin{proposition}[Synaptic Gradient]
\label{prop:grad_gsyn}
\begin{equation}
\frac{\partial V_n}{\partial g_i^{\mathrm{syn}}} = x_i R_n^{\mathrm{tot}} (E_i - V_n)
\label{eq:grad_gsyn}
\end{equation}
\end{proposition}
\begin{proof}
Apply quotient rule to \eqref{eq:voltage}:
\begin{align*}
\frac{\partial V_n}{\partial g_i^{\mathrm{syn}}} &= \frac{E_i x_i \cdot g_n^{\mathrm{tot}} - \left(\sum_j E_j x_j g_j^{\mathrm{syn}} + \sum_j V_j g_j^{\mathrm{den}}\right) \cdot x_i}{(g_n^{\mathrm{tot}})^2} \\
&= \frac{E_i x_i}{g_n^{\mathrm{tot}}} - \frac{V_n x_i}{g_n^{\mathrm{tot}}} = x_i R_n^{\mathrm{tot}} (E_i - V_n). \qedhere
\end{align*}
\end{proof}

\begin{proposition}[Child Voltage Gradient]
\label{prop:grad_V}
\begin{equation}
\frac{\partial V_n}{\partial V_i} = g_i^{\mathrm{den}} R_n^{\mathrm{tot}}
\label{eq:grad_V}
\end{equation}
\end{proposition}

\begin{proposition}[Dendritic Gradient]
\label{prop:grad_gden}
\begin{equation}
\frac{\partial V_n}{\partial g_i^{\mathrm{den}}} = R_n^{\mathrm{tot}} (V_i - V_n)
\label{eq:grad_gden}
\end{equation}
\end{proposition}

\begin{proposition}[Additional Local Sensitivities]
\label{prop:additional}
\[
\frac{\partial V_n}{\partial x_i} = g_i^{\mathrm{syn}} R_n^{\mathrm{tot}} (E_i - V_n), 
\qquad 
\frac{\partial V_n}{\partial E_i} = x_i g_i^{\mathrm{syn}} R_n^{\mathrm{tot}},
\qquad
\frac{\partial V_n}{\partial g^{\mathrm{leak}}} = - V_n R_n^{\mathrm{tot}}.
\]
\end{proposition}

\begin{remark}
The sensitivity $\partial V_n/\partial g^{\mathrm{leak}} = - V_n R_n^{\mathrm{tot}}$ applies if $g^{\mathrm{leak}}$ is a trainable parameter. In all reported experiments, we fix $g^{\mathrm{leak}}{=}1$ (normalized units).
\end{remark}

\subsection{Shunting Inhibition and Divisive Gain Control}
\label{sec:shunting}

A conductance-based inhibitory synapse with $E_{\mathrm{inh}}\approx E_\mathrm{leak}{=}0$ contributes current $I_{\mathrm{inh}} = (0 - V_n)\, x_j g_j^{\mathrm{syn}}$ and increases $g_n^{\mathrm{tot}}$ in \eqref{eq:conductance}. 

\begin{proposition}[Subthreshold Effect of Shunts]
For a pure shunt ($E_j=0$), the steady-state sensitivity to the inhibitory conductance is
\[
\frac{\partial V_n}{\partial g_j^{\mathrm{syn}}}
= x_j R_n^{\mathrm{tot}} (0 - V_n) = - x_j R_n^{\mathrm{tot}} V_n.
\]
Thus $V_n$ is multiplicatively attenuated (divisive normalization) by increased inhibitory conductance at fixed drives.
\end{proposition}

\begin{remark}[Divisive vs.\ Subtractive at the Firing-Rate Level]
While shunting produces divisive scaling of subthreshold voltages, its net effect on firing rates can be subtractive in many regimes \cite{holt1997shunting}, so we report both voltage- and rate-level analyses in experiments. Normalization via added conductance is consistent with canonical divisive normalization models in cortex \cite{carandini2012normalization}.
\end{remark}

\paragraph{Inhibitory/shunting synapses.}
For an inhibitory synapse with $E_j \approx 0$, the 3F update reduces to
\[
\Delta g_{j,\mathrm{inh}}^{\mathrm{syn}} 
= \eta\, \langle x_j R_n^{\mathrm{tot}} (-V_n)\, e_n \rangle_B,
\]
i.e., anti-Hebbian in $V_n$ and divisive in $g_n^{\mathrm{tot}}$. 
With 4F/5F, multiply by $\rho$ and $\phi$ (Def.~\ref{def:phi_fixed}). Note that the same multiplicative factors are applied to both excitatory and inhibitory synapses in the implementation; the sign difference arises solely from the driving force $(E_j - V_n)$.

\subsection{Loss Propagation}

Let $V_0$ denote the somatic/output compartment. The decoder produces $\hat y = W_{\mathrm{dec}} V_0$ (linear case), and $L$ is the task loss. Define the error gradients:
\begin{equation}
\delta^y := \frac{\partial L}{\partial \hat y}, \qquad
\delta_0 := \frac{\partial L}{\partial V_0} = \left(\frac{\partial \hat y}{\partial V_0}\right)^\top \delta^y
= W_{\mathrm{dec}}^\top \delta^y.
\end{equation}

\begin{theorem}[Backpropagation on a Dendritic Tree]
\label{thm:tree_backprop}
Let the dendritic morphology be a rooted tree with soma/output at node $0$. For any compartment $n$ with parent set $\mathcal{P}(n)$ (typically $|\mathcal{P}(n)|{=}1$), the loss gradient satisfies the recursion
\begin{equation}
\frac{\partial L}{\partial V_n}
=\sum_{p \in \mathcal{P}(n)} \frac{\partial L}{\partial V_p}\, \frac{\partial V_p}{\partial V_n}
=\sum_{p \in \mathcal{P}(n)} \delta_p \, R_p^{\mathrm{tot}}\, g_{n\to p}^{\mathrm{den}},
\qquad \delta_p \equiv \frac{\partial L}{\partial V_p}.
\label{eq:tree_recursion}
\end{equation}
Unrolling the recursion yields a sum over all directed paths $\mathcal{P}: n \leadsto 0$:
\begin{equation}
\frac{\partial L}{\partial V_n} 
= \frac{\partial L}{\partial V_0}
\sum_{\mathcal{P}:n\leadsto 0}
\prod_{(i\to k)\in \mathcal{P}} R_k^{\mathrm{tot}}\, g_{i\to k}^{\mathrm{den}}.
\label{eq:path_sum}
\end{equation}
\end{theorem}
\begin{proof}
Apply the multivariate chain rule on the directed acyclic computation graph defined by the tree; use Proposition~\ref{prop:grad_V}. Each path contributes a product of edge sensitivities. Summing over parents produces \eqref{eq:tree_recursion}; unrolling yields \eqref{eq:path_sum}.
\end{proof}

\begin{corollary}[Chain Case]
\label{cor:chain}
If the morphology is a simple chain $V_0 \leftarrow V_1 \leftarrow \cdots \leftarrow V_n$, \eqref{eq:path_sum} reduces to
\begin{equation}
\frac{\partial L}{\partial V_n} = \frac{\partial L}{\partial V_0} \prod_{i=1}^n R_{i-1}^{\mathrm{tot}} g_i^{\mathrm{den}}.
\label{eq:exact_backprop}
\end{equation}
Defining $g_0^{\mathrm{den}} = 1$ and reindexing:
\begin{equation}
\frac{\partial L}{\partial V_n} = \frac{\partial L}{\partial V_0} \prod_{i=0}^n R_i^{\mathrm{tot}} g_i^{\mathrm{den}}.
\label{eq:exact_backprop_reindex}
\end{equation}
\end{corollary}

\begin{corollary}[Synaptic Parameter Gradient]
\begin{equation}
\frac{\partial L}{\partial g_j^{\mathrm{syn}}} = \frac{\partial L}{\partial V_0} \left(\prod_{i=0}^n R_i^{\mathrm{tot}} g_i^{\mathrm{den}}\right) x_j (E_j - V_n)
\label{eq:exact_gsyn}
\end{equation}
\end{corollary}

\begin{corollary}[Dendritic Parameter Gradient]
\begin{equation}
\frac{\partial L}{\partial g_j^{\mathrm{den}}} = \frac{\partial L}{\partial V_0} \left(\prod_{i=0}^n R_i^{\mathrm{tot}} g_i^{\mathrm{den}}\right) (V_j - V_n)
\label{eq:exact_gden}
\end{equation}
\end{corollary}

\section{Local Learning Approximations}

\subsection{Broadcast Error Approximation}

\begin{definition}[Local Approximation]
Replace the exact gradient $\frac{\partial L}{\partial V_n}$ with a broadcast error signal $e_n$ derived from the output error $\delta_0 = \frac{\partial L}{\partial V_0}$:
\begin{equation}
\frac{\partial L}{\partial V_n} \approx e_n, \qquad \prod_{i=0}^n R_i^{\mathrm{tot}} g_i^{\mathrm{den}} \approx 1
\label{eq:local_approx}
\end{equation}
\end{definition}

Three broadcast modes are considered:

\paragraph{(A) Scalar broadcast.}
For minibatch index $b$:
\begin{equation}
\bar{\delta}(b) = \frac{1}{d_{\mathrm{out}}} \sum_{k=1}^{d_{\mathrm{out}}} \delta_k(b), \qquad e_n(b) = \bar{\delta}(b) \mathbf{1}_{d_n}
\end{equation}

\paragraph{(B) Per-compartment mapping.}
If $d_n = d_{\mathrm{out}}$: $e_n(b) = \delta(b)$. Otherwise, \emph{fallback to scalar broadcast}. An optional DFA-style mode uses a fixed random feedback matrix $B_n \in \mathbb{R}^{d_n \times d_{\mathrm{out}}}$ sampled once at initialization: $e_n(b) = B_n \delta(b)$. This supports testing Theorem~\ref{thm:fa_alignment}.

\paragraph{(C) Local mismatch modulation.}
Let $P_n(b)$ be parent compartment drive (e.g., blocklinear output). Define centered mismatch:
\begin{equation}
\varepsilon_n(b) = \left(P_n(b) - V_n(b)\right) - \frac{1}{B} \sum_{t=1}^B \left(P_n(t) - V_n(t)\right)
\end{equation}
Then:
\begin{equation}
e_n(b) = \bar{\delta}(b) \varepsilon_n(b)
\end{equation}

\subsection{Gradient Alignment with Broadcast Errors}
\label{sec:alignment}

Define the exact synaptic gradient at layer $n$ by $g^{\mathrm{exact}} = \delta_0 \cdot \Xi_n$, where $\Xi_n$ collects local factors and the exact path-sum \eqref{eq:path_sum}. The local 3F gradient with broadcast error $e_n=B_n \delta_0$ is $g^{\mathrm{local}} = e_n \cdot \widehat{\Xi}_n$, where $\widehat{\Xi}_n$ omits the path-sum.

\begin{theorem}[Positive Expected Alignment under Random Broadcast]
\label{thm:fa_alignment}
Let $B_n\in\mathbb{R}^{d_n\times d_{\mathrm{out}}}$ have i.i.d.\ zero-mean entries with $\mathbb{E}[B_n^\top B_n]=\alpha I$. If the decoder aligns with the forward pathway (standard during training), then
\[
\mathbb{E}\big[\cos\angle(g^{\mathrm{local}},g^{\mathrm{exact}})\big] \ \ge\ c_n>0,
\]
where $c_n$ depends on $\alpha$ and the average correlation between $\widehat{\Xi}_n$ and $\Xi_n$. Thus $g^{\mathrm{local}}$ provides a descent direction in expectation.
\end{theorem}
\begin{proof}[Sketch]
Adapt the feedback-alignment argument \cite{lillicrap2016random,nokland2016dfa}: fixed random feedback suffices for alignment as forward weights adapt. Here, $\widehat{\Xi}_n$ is proportional to $\Xi_n$ up to the missing path factor; Jensen bounds on \eqref{eq:path_sum} yield $c_n>0$.
\end{proof}

\subsection{Three-Factor Rule (3F)}

\begin{definition}[3F Learning Rule]
For synaptic conductances:
\begin{equation}
\Delta g_j^{\mathrm{syn}} = \eta \left\langle x_j R_n^{\mathrm{tot}} (E_j - V_n) e_n \right\rangle_B
\label{eq:3f_syn}
\end{equation}
For dendritic conductances:
\begin{equation}
\Delta g_j^{\mathrm{den}} = \eta \left\langle R_n^{\mathrm{tot}} (V_j - V_n) e_n \right\rangle_B
\label{eq:3f_den}
\end{equation}
where $\langle \cdot \rangle_B$ denotes batch average.
\end{definition}

\begin{remark}
The three factors are: (1) presynaptic activity $x_j$ or voltage difference $(V_j - V_n)$, (2) postsynaptic modulation $(E_j - V_n)$ or $R_n^{\mathrm{tot}}$, (3) broadcast error $e_n$.
\\[2pt]
\textbf{Symmetry note.} The same multiplicative factors ($R_n^{\mathrm{tot}}$, $\rho$, $\phi$, $s_j$) apply to both excitatory and inhibitory synapses; the sign difference arises solely from the driving force $(E_j - V_n)$.
\end{remark}

\subsection{Four-Factor Rule (4F): Morphology Correlation}

\begin{definition}[Morphology Factor]
Let $\bar{V}_n = \frac{1}{d_n} \sum_{j=1}^{d_n} V_{n,j}$ be the mean voltage over compartments in layer $n$. Define the correlation with output:
\begin{equation}
\rho_n = \frac{\Cov(\bar{V}_n, \bar{V}_0)}{\sqrt{\Var(\bar{V}_n) \Var(\bar{V}_0)} + \varepsilon}
\label{eq:rho}
\end{equation}
\end{definition}

\begin{proposition}[4F Update Rule]
Multiply 3F updates by $\rho_n$:
\begin{align}
\Delta g_j^{\mathrm{syn}} &= \eta \rho_n \left\langle x_j R_n^{\mathrm{tot}} (E_j - V_n) e_n \right\rangle_B \label{eq:4f_syn} \\
\Delta g_j^{\mathrm{den}} &= \eta \rho_n \left\langle R_n^{\mathrm{tot}} (V_j - V_n) e_n \right\rangle_B \label{eq:4f_den}
\end{align}
\end{proposition}

\begin{proposition}[Approximate Gradient Alignment]
Let $L$ be a smooth loss. If layer $n$ contributes to the output primarily through its mean activity, then
$\mathbb{E}[\frac{\partial L}{\partial \bar{V}_n} \cdot \bar{V}_n] \propto \rho_n \cdot \Var(\bar{V}_n)$.
Thus $\rho_n$ weights updates by the layer's relevance to the task.
\end{proposition}

\paragraph{Estimators (EMA / online).}
For minibatches $B\ge 2$, estimate $\rho_n$ from \eqref{eq:rho} with an EMA over batches.
For $B=1$ (online), maintain means $\mu_x,\mu_y$, variances $\sigma_x^2,\sigma_y^2$, and covariance $C_{xy}$ for
$x_t=\bar V_0^{(t)}$ and $y_t=\bar V_n^{(t)}$ using Welford's numerically stable algorithm \cite{welford1962note}:
\begin{align}
\mu_x^{(t)} &= (1-\alpha)\mu_x^{(t-1)} + \alpha x_t, &
\mu_y^{(t)} &= (1-\alpha)\mu_y^{(t-1)} + \alpha y_t,\nonumber\\
\delta_x &= x_t - \mu_x^{(t-1)}, &
\delta_y &= y_t - \mu_y^{(t-1)},\nonumber\\
\sigma_x^{2(t)} &= (1-\alpha)\sigma_x^{2(t-1)} + \alpha\, \delta_x^2, &
\sigma_y^{2(t)} &= (1-\alpha)\sigma_y^{2(t-1)} + \alpha\, \delta_y^2,\nonumber\\
C_{xy}^{(t)} &= (1-\alpha)C_{xy}^{(t-1)} + \alpha\, \delta_x \delta_y.
\end{align}
Then $\rho_n^{(t)} = C_{xy}^{(t)} / (\sqrt{\sigma_x^{2(t)}\sigma_y^{2(t)}}+\varepsilon)$, where $\alpha$ is the EMA rate.

\subsection{Five-Factor Rule (5F): Conditional Information}

\begin{definition}[Conditional Predictability Factor]
\label{def:phi_fixed}
Let $P_n$ be parent compartment voltage. Define the conditional variance via ridge regression:
\begin{align}
\beta_n &= \frac{\Cov(V_n, P_n)}{\Var(P_n) + \lambda} \label{eq:beta} \\
\sigma_{\mathrm{res}}^2 &= \Var(V_n) - \beta_n \Cov(V_n, P_n) \label{eq:residual_var}
\end{align}
The information proxy is:
\begin{equation}
\phi_n = \frac{\Var(V_n)}{\sigma_{\mathrm{res}}^2 + \varepsilon} = \frac{1}{1 - R_n^2} \geq 1,
\label{eq:phi}
\end{equation}
where $R_n^2=\frac{\beta_n\,\Cov(V_n,P_n)}{\Var(V_n)}$ is the (ridge) coefficient of determination.
\end{definition}

\begin{remark}[Information-Theoretic Interpretation]
$\phi_n$ increases when $V_n$ is \emph{more} predictable from its parent $P_n$ (higher $R^2$), amplifying updates for compartments with strong signal propagation. In practice, $\phi_n$ is clamped to $[0.25, 4.0]$ for stability. An alternative $\phi_n = 1 - R^2$ would instead emphasize compartments with unique information; both are valid depending on whether coherent signal flow or novelty is prioritized.
\end{remark}

\begin{proposition}[5F Update Rule]
\begin{align}
\Delta g_j^{\mathrm{syn}} &= \eta \rho_n \phi_n \left\langle x_j R_n^{\mathrm{tot}} (E_j - V_n) e_n \right\rangle_B \label{eq:5f_syn} \\
\Delta g_j^{\mathrm{den}} &= \eta \rho_n \phi_n \left\langle R_n^{\mathrm{tot}} (V_j - V_n) e_n \right\rangle_B \label{eq:5f_den}
\end{align}
\end{proposition}

\section{Morphology-Aware Extensions}

Standard 4F/5F rules use layer-wise factors $\rho_n$, $\phi_n$ that ignore branch-specific topology. We introduce four extensions that explicitly incorporate dendritic tree structure.

\subsection{Path-Integrated Propagation}

\begin{definition}[Path Factor]
Define recursively:
\begin{equation}
\pi_n = \begin{cases}
1 & n = 0 \\
\pi_{n-1} \cdot R_{n-1}^{\mathrm{tot}} \cdot \bar{g}_{n-1}^{\mathrm{den}} & n \geq 1
\end{cases}
\label{eq:path_factor}
\end{equation}
where $\bar{g}_{n-1}^{\mathrm{den}}$ is the mean dendritic conductance from layer $n-1$ to $n$.
\end{definition}

\begin{proposition}[Exact for Chains]
\label{prop:path_exact_chain}
For a chain morphology (single path), the path factor \eqref{eq:path_factor} satisfies
\begin{equation}
\pi_n = \prod_{i=0}^{n-1} R_i^{\mathrm{tot}} g_i^{\mathrm{den}}
\end{equation}
and thus exactly matches \eqref{eq:path_sum}.
\end{proposition}

\begin{remark}[Sandwich bounds for trees]
\label{rem:path_tree_bounds}
For a tree, let $\mathcal{P}$ be the set of directed paths from $n$ to $0$ and define, at each depth $d$,
$m_d := \min_{\mathcal{P}} a_{d,\mathcal{P}}$ and $M_d := \max_{\mathcal{P}} a_{d,\mathcal{P}}$, where
$a_{d,\mathcal{P}}$ is the edge factor $R_k^{\mathrm{tot}} g_{i\to k}^{\mathrm{den}}$ at depth $d$ along path $\mathcal{P}$.
Then
\[
|\mathcal{P}| \prod_{d} m_d
\ \le\ 
\sum_{\mathcal{P}} \prod_{d} a_{d,\mathcal{P}}
\ \le\
|\mathcal{P}| \prod_{d} M_d.
\]
If per-depth factors are narrowly concentrated ($m_d\!\approx\!M_d$), replacing the sum by a product of per-depth means (our $\pi_n$) is accurate. Modulating the error by $\pi_n$ yields $e_n^{\mathrm{path}} = e_n \cdot \pi_n$, which better approximates exact backpropagation.
\end{remark}

\begin{corollary}[Depth Attenuation]
From Lemma~\ref{lem:convex}, $R_k^{\mathrm{tot}} g^{\mathrm{den}}_{i\to k} < 1$. Therefore any product $\prod Rg$ in \eqref{eq:path_sum} decays exponentially with depth, motivating path-based error attenuation.
\end{corollary}

\begin{remark}
In practice, $\pi_n$ is computed as a \emph{per-sample scalar} (broadcast to all compartments in layer $n$), using the arithmetic mean of outgoing dendritic conductances at each depth. The modulated error is $e_n \leftarrow e_n \cdot \pi_n$.
\end{remark}

\paragraph{Theoretical effect.}
Path propagation introduces depth-dependent attenuation: deeper compartments receive exponentially smaller error signals $e_n \sim \prod R_i g_i$. This encourages specialization: shallow layers learn direct input-output mappings, while deep layers integrate over longer paths. In practice, we use a \emph{per-sample scalar} path factor $\pi_n(b)$ for stability and consistent broadcasting across layers with different widths.

\subsection{Additional Extensions}

Beyond path propagation, we implement three further morphology-aware mechanisms (detailed in Appendix~\ref{app:morphology}):
\begin{itemize}
\item \textbf{Depth modulation:} Per-branch scaling $\rho_j = \rho_{\mathrm{base}} / (d_j + \alpha)$ that mirrors cable attenuation, biasing learning toward proximal synapses.
\item \textbf{Dendritic normalization:} Update normalization by total branch conductance, $\Delta g_j^{\mathrm{den}} \leftarrow \Delta g_j^{\mathrm{den}} / (\sum_k g_k^{\mathrm{den}} + \varepsilon)$, which stabilizes update variance analogous to homeostatic scaling \cite{turrigiano2008homeostatic}.
\item \textbf{Apical/basal differentiation:} Branch-type-specific scaling factors $s_j$ that allow differential plasticity in feedback (apical) vs.\ feedforward (basal) compartments \cite{larkum2013apical}.
\end{itemize}

\paragraph{HSIC auxiliary objectives.}
We optionally add Hilbert-Schmidt Independence Criterion (HSIC) losses \cite{gretton2005hsic} as auxiliary objectives: a self-decorrelation term that encourages diverse representations within each layer, and a target-correlation term that aligns layer activations with labels. Moderate HSIC weights improve performance on context-gating tasks while having negligible effect on MNIST (see Section~\ref{sec:experiments}). Full definitions and gradients are in Appendix~\ref{app:hsic}.

\section{Implementation Details}
\label{sec:implementation}

\subsection{Units and Normalization}

\begin{table}[h]\centering
\begin{tabular}{@{}lll@{}}\toprule
Quantity & Symbol & Typical units (scaled)\\\midrule
Voltage & $V$ & mV (normalized to $[-1,1]$)\\
Synaptic conductance & $g^{\mathrm{syn}}$ & nS (nonnegative)\\
Dendritic conductance & $g^{\mathrm{den}}$ & nS (nonnegative)\\
Leak conductance & $g^{\mathrm{leak}}$ & nS (set to $1$ in normalized units)\\
Input resistance & $R^{\mathrm{tot}}$ & $\mathrm{nS}^{-1}$ (normalized $\le 1$)\\\bottomrule
\end{tabular}
\caption{Units and normalization conventions.}
\label{tab:units}
\end{table}

\subsection{Positive Weight Parameterization}

To enforce $g \geq 0$, we can use exponential:
\begin{equation}
g = \exp(\theta), \quad \theta \in \mathbb{R}
\end{equation}
Chain rule for gradients:
\begin{equation}
\frac{\partial L}{\partial \theta} = \frac{\partial L}{\partial g} \cdot g
\end{equation}
Alternatively, use softplus $g = \log(1+\exp(\theta))$ to avoid extreme gradients.

\subsection{Decoder Update Modes}

Let $W_{\mathrm{dec}}$ map $V_L \to y \in \mathbb{R}^{d_{\mathrm{out}}}$. Three modes:
\begin{enumerate}
\item \textbf{Backprop}: $\nabla_{W_{\mathrm{dec}}} L$ via autograd.
\item \textbf{Local}: $\Delta W_{\mathrm{dec}} = \eta \langle \delta_0 V_L^\top \rangle_B$ (3-factor).
\item \textbf{Frozen}: $\Delta W_{\mathrm{dec}} = 0$.
\end{enumerate}

\subsection{Algorithm Summary}

\begin{algorithm}[H]
\caption{Local Credit Assignment with Morphology-Aware Extensions}
\begin{algorithmic}[1]
\STATE \textbf{Input:} Model, minibatch $(x, y)$, config $\mathcal{C}$
\STATE Forward pass: $\hat{y} = f(x; \mathbf{g}^{\mathrm{syn}}, \mathbf{g}^{\mathrm{den}})$
\STATE Compute loss $L$ and output error $\delta^{y} = \frac{\partial L}{\partial \hat{y}}$
\STATE Compute somatic error $\delta_0 = W_{\mathrm{dec}}^\top \delta^{y}$
\FOR{each layer $n$ (reverse order)}
    \STATE Broadcast error: $e_n = \text{broadcast}(\delta_0, \mathcal{C})$
    \IF{path propagation enabled}
        \STATE Compute $\pi_n$ via \eqref{eq:path_factor}; $e_n \leftarrow e_n \cdot \pi_n$
    \ENDIF
    \STATE Compute $\rho_n$ via \eqref{eq:rho}
    \IF{depth modulation enabled}
        \STATE $\rho_n \leftarrow [\rho_1, \ldots, \rho_{d_n}]$ via \eqref{eq:rho_depth}
    \ENDIF
    \STATE Compute $\phi_n$ via \eqref{eq:phi}
    \STATE Compute branch scales $s_j$ via \eqref{eq:branch_scale}
    \STATE Synaptic updates: $\Delta g_j^{\mathrm{syn}} = \eta s_j \rho_j \phi_n \langle x_j R_n^{\mathrm{tot}} (E_j - V_n) e_n \rangle_B$
    \STATE Dendritic updates: $\Delta g_j^{\mathrm{den}} = \eta s_j \rho_j \phi_n \langle R_n^{\mathrm{tot}} (V_j - V_n) e_n \rangle_B$
    \IF{dendritic normalization enabled}
        \STATE $\Delta g_j^{\mathrm{den}} \leftarrow \Delta g_j^{\mathrm{den}} / (G_n + \varepsilon)$ via \eqref{eq:dendritic_norm}
    \ENDIF
    \IF{HSIC enabled}
        \STATE Add HSIC gradients to $\Delta g_j^{\mathrm{syn}}$
    \ENDIF
\ENDFOR
\STATE Clip gradients; optimizer step
\end{algorithmic}
\end{algorithm}


\section{Theoretical Comparison}

\begin{table}[h]
\centering
\begin{tabular}{@{}llcl@{}}
\toprule
\textbf{Method} & \textbf{Factors} & \textbf{Complexity} & \textbf{Best observed regime (current sweeps)} \\
\midrule
3F & $x, (E-V), e$ & $\mathcal{O}(1)$ & Baseline local plasticity; weak on contextual/hierarchical tasks \\
4F & 3F $+ \rho$ & $\mathcal{O}(1)$ & Better scaling/conditioning than 3F; still limited performance ceiling \\
5F & 4F $+ \phi$ & $\mathcal{O}(d_n)$ & Strongest overall local competence (MNIST, context gating) \\
\midrule
5F + Path & 5F $+ \pi$ & $\mathcal{O}(L)$ & Strongest impact on representation metrics; selective accuracy gains \\
5F + Depth & 5F, $\rho \to \rho_j$ & $\mathcal{O}(d_n)$ & Useful in deeper/branched morphologies \\
5F + Norm & 5F + normalization & $\mathcal{O}(d_n)$ & Stabilizes update scale across branches \\
5F + Types & 5F $\times s_j$ & $\mathcal{O}(1)$ & Tests apical/basal specialization hypotheses \\
\bottomrule
\end{tabular}
\caption{Variant taxonomy: computational cost and current empirical regime map ($L$ = depth, $d_n$ = compartments).}
\end{table}

\begin{table}[h]
\centering
\small
\begin{tabular}{@{}lll@{}}
\toprule
\textbf{Component} & \textbf{Biological Analog} & \textbf{Key Result} \\
\midrule
Conductance scaling $R_n^{\mathrm{tot}}$ & Input resistance & Lemma~\ref{lem:convex}: $0 < R_n^{\mathrm{tot}} \le 1$ \\
Driving force $(E_j - V_n)$ & Synaptic current & Prop.~\ref{prop:grad_gsyn}: Local sensitivity \\
Shunting inhibition & Divisive normalization & Sec.~\ref{sec:shunting}: $\partial V/\partial g_{\text{inh}} \propto -V$ \\
Path factor $\pi_n$ & Cable attenuation & Prop.~\ref{prop:path_exact_chain}: Depth decay \\
Morphology factor $\rho_n$ & Layer correlation & Eq.~\eqref{eq:rho}: Task relevance \\
Information factor $\phi_n$ & Conditional predictability & Eq.~\eqref{eq:phi}: $1/(1-R^2)$ \\
Dendritic normalization & Homeostatic scaling & Sec.~4.3: Variance stabilization \\
Branch-type scaling & Apical vs.\ basal & Sec.~4.4: Compartment specialization \\
Broadcast alignment & Feedback alignment & Thm.~\ref{thm:fa_alignment}: $\mathbb{E}[\cos \angle] > 0$ \\
\bottomrule
\end{tabular}
\caption{Summary of theoretical components and their biological/algorithmic interpretations.}
\end{table}

\section{Experiments}
\label{sec:experiments}

\subsection{Setup and Main Claims}

We evaluate local credit assignment rules through a structured empirical program that separates (i)~model capacity, (ii)~local-rule competence, and (iii)~mechanistic regime dependence. Our main findings are:
\begin{enumerate}
\item \textbf{Shunting is the key regime:} Shunting dendritic cores outperform additive controls under the same local rule, with the gap widening under stronger inhibition (Figs.~\ref{fig:shunting_regime}, \ref{fig:claimA_shunting_heatmap}).
\item \textbf{Gradient fidelity explains the advantage:} Shunting models produce local gradients with dramatically better alignment to exact backpropagation (Table~\ref{tab:gradient_alignment_summary}, Fig.~\ref{fig:gradient_alignment_components}).
\item \textbf{5F with per-soma broadcast is the best local rule:} It reaches $0.914$ test accuracy on MNIST ($0.965$ backprop ceiling) and $0.803$ on context gating ($0.864$ ceiling).
\end{enumerate}

\subsection{Phase-Based Evaluation (Phase 1--Phase 3)}

We use a phase-based sweep suite:
\begin{itemize}
\item \textbf{Phase 1 (capacity ceiling):} train identical architectures with standard backprop to establish a meaningful accuracy ceiling per dataset and core type (Fig.~\ref{fig:phase1_capacity}).
\item \textbf{Phase 2 (local competence):} within the Phase-1 capacity regime, sweep LocalCA knobs (rule variant, broadcast mode, morphology-aware modulators, HSIC auxiliaries) to find competent local configurations.
\item \textbf{Phase 3 (mechanistic claims):} targeted sweeps testing regime dependence (shunting vs additive across inhibition/noise), morphology scaling, error shaping, and information analyses.
\end{itemize}

\subsection{Variant Ranking Across Local Rule Families}

To make the rule comparison explicit, we aggregate completed Phase-2 and Phase-2b runs on MNIST and context gating and rank 3F/4F/5F under matched local-learning sweeps.

\begin{table}[t]
\centering
\small
\begin{tabular}{@{}llcc@{}}
\toprule
\textbf{Dataset} & \textbf{Rule} & \textbf{Top-10 valid mean} & \textbf{Top-10 test mean} \\
\midrule
MNIST & 3F & 0.611 & 0.622 \\
MNIST & 4F & 0.620 & 0.628 \\
MNIST & 5F & \textbf{0.912} & \textbf{0.916} \\
\midrule
Context gating & 3F & 0.398 & 0.396 \\
Context gating & 4F & 0.411 & 0.411 \\
Context gating & 5F & \textbf{0.807} & \textbf{0.789} \\
\bottomrule
\end{tabular}
\caption{\textbf{Rule-family ranking from completed local-competence sweeps.} Values are averages over the top 10 runs (ranked by validation accuracy) within each dataset and rule family. 5F dominates 3F/4F in both datasets under the current sweep space.}
\label{tab:variant_ranking_phase2}
\end{table}

\paragraph{Broadcast-mode interaction.}
Per-soma broadcast strongly outperforms scalar and local-mismatch modes for both datasets (MNIST: $0.918$ vs $0.889$ vs $0.200$ test; context gating: $0.827$ vs $0.699$ vs $0.114$ test). This motivates treating the broadcast mode as a primary experimental factor.

\subsection{Results}

\paragraph{Capacity ceilings (standard backprop).}
Standard backprop achieves $0.978$ test accuracy on MNIST (point MLP) and $0.965$ (dendritic shunting), $0.864$ on context gating (dendritic shunting), and $\approx 0.49$ on CIFAR-10 with flattened inputs (Fig.~\ref{fig:phase1_capacity}). These ceilings define the reference for local-rule comparisons.

\paragraph{Local competence.}
With the best local configuration (5F rule, per-soma broadcast, dendritic shunting with layer size $128$, branch factors $[3,3]$, 20 inhibitory synapses), local credit assignment reaches $0.914\pm0.003$ test accuracy on MNIST and $0.803\pm0.006$ on context gating (with moderate HSIC weight), compared to backprop ceilings of $0.965$ and $0.864$ respectively (Fig.~\ref{fig:phase2b_hsic_weight}).

\paragraph{Regime dependence.}
The shunting advantage is largest in high-inhibition regimes: per-soma broadcast with shunting shows $\approx 0.19$ test-accuracy advantage over additive controls at higher inhibition levels. Representation-level metrics (mutual information) show clear broadcast/path interactions (Fig.~\ref{fig:phase_information_panel}).

\paragraph{Baseline comparison on matched architectures.}
To contextualize local learning performance against alternative biologically plausible algorithms, we compare backpropagation, local rules (5F, 3F), and direct feedback alignment (DFA) on MNIST using matched small architectures (Fig.~\ref{fig:main_results}). Backpropagation on the shunting dendritic core achieves $0.84$ test accuracy, while local 5F and 3F rules reach $\approx 0.38$---substantially above the additive control ($\approx 0.11$), confirming shunting as the key enabler. DFA achieves $0.34$ on the shunting core but matches backpropagation on point MLP ($0.47$), consistent with its known strength in sequential architectures. Notably, feedback alignment (FA) fails entirely on dendritic architectures due to their non-sequential topology while DFA works across all architectures.

\begin{figure}[t]
\centering
\includegraphics[width=0.98\linewidth]{fig_main_results.pdf}
\caption{\textbf{Training strategy comparison on MNIST (matched architecture).} Test accuracy for backpropagation, local rules (5F, 3F with per-soma broadcast), and DFA across three architectures. Shunting dendritic cores provide the strongest backprop ceiling ($0.84$) and the largest local-rule advantage over additive controls. DFA performs comparably to local rules on dendritic architectures but matches backprop on the point MLP. Error bars: 95\% CI.}
\label{fig:main_results}
\end{figure}

\paragraph{Learning dynamics.}
Figure~\ref{fig:learning_curves} shows learning curves across training strategies. Backpropagation converges fastest, with the shunting dendritic core reaching high accuracy within 10 epochs. Local rules converge more slowly but steadily improve throughout training. On Fashion-MNIST, local 5F achieves the strongest performance on shunting cores ($\approx 0.36$), maintaining the shunting advantage observed on MNIST.

\begin{figure*}[t]
\centering
\includegraphics[width=0.95\textwidth]{fig_learning_curves.pdf}
\caption{\textbf{Learning curves across training strategies.} (A)~MNIST with backpropagation and DFA baselines, showing faster convergence for backprop on shunting cores. (B)~MNIST with local rules (3F/4F/5F), comparing shunting, additive, and MLP architectures. (C)~Fashion-MNIST with local rules, confirming the shunting advantage generalizes to a harder dataset. Shaded regions: $\pm 1$ s.d.\ across seeds.}
\label{fig:learning_curves}
\end{figure*}

\paragraph{Fashion-MNIST generalization.}
Local rules transfer to Fashion-MNIST (Fig.~\ref{fig:fashion_mnist}), a substantially harder variant. With per-soma broadcast, shunting cores reach $\approx 0.35$ test accuracy across all three rule variants (5F: $0.358$, 4F: $0.352$, 3F: $0.344$), while additive controls collapse to chance ($\approx 0.10$). The shunting advantage ($\approx 3.5\times$) is even larger than on MNIST, suggesting that shunting-based local credit assignment becomes \emph{more} important as task difficulty increases.

\begin{figure}[t]
\centering
\includegraphics[width=0.98\linewidth]{fig_fashion_mnist_results.pdf}
\caption{\textbf{Fashion-MNIST local learning results.} (A)~Test accuracy by rule variant and architecture (per-soma broadcast). Shunting cores outperform additive controls by $\approx 3.5\times$. (B)~MNIST vs Fashion-MNIST comparison on shunting cores with per-soma broadcast. Both datasets show similar absolute performance under local rules, with Fashion-MNIST marginally lower. Error bars: 95\% CI.}
\label{fig:fashion_mnist}
\end{figure}

\subsection{Gradient-Fidelity Diagnostic (Local vs Backprop)}

To test whether improved performance corresponds to better credit signals, we compare LocalCA and backprop gradients on the \emph{same batch and same initial weights}, component-wise. For each parameter tensor \(p\), we compute
\begin{equation}
\cos_p = \frac{\langle g_p^{\mathrm{local}}, g_p^{\mathrm{bp}} \rangle}{\|g_p^{\mathrm{local}}\|_2 \, \|g_p^{\mathrm{bp}}\|_2 + \varepsilon},
\end{equation}
and a scale mismatch
\begin{equation}
\Delta^{\mathrm{scale}}_p = \left| \log_{10} \frac{\|g_p^{\mathrm{local}}\|_2}{\|g_p^{\mathrm{bp}}\|_2 + \varepsilon} \right|.
\end{equation}
We then aggregate by parameter count across component groups (excitatory synapses, inhibitory synapses, dendritic conductances, reactivation).

\begin{table}[t]
\centering
\small
\begin{tabular}{@{}llccc@{}}
\toprule
\textbf{Dataset} & \textbf{Core} & \textbf{Weighted cosine} & \textbf{Scale mismatch} & \textbf{Rel.\ L2} \\
\midrule
MNIST & Shunting & \textbf{0.202} & \textbf{0.117} & \textbf{1.130} \\
MNIST & Additive & 0.006 & 1.053 & 13.324 \\
\midrule
Context gating & Shunting & \textbf{0.108} & \textbf{0.036} & \textbf{1.404} \\
Context gating & Additive & -0.007 & 2.154 & 145.965 \\
\bottomrule
\end{tabular}
\caption{\textbf{Gradient-fidelity summary (5F + per-soma broadcast).} Local vs backprop gradients compared on matched weights and batches. ``Weighted cosine'' is parameter-count weighted over component groups. ``Scale mismatch'' is $|\log_{10}(\|g^{\mathrm{local}}\|/\|g^{\mathrm{bp}}\|)|$ (lower is better). Shunting networks show $30\times$ better directional alignment and $10\times$ lower scale distortion than additive controls.}
\label{tab:gradient_alignment_summary}
\end{table}

\begin{figure}[t]
\centering
\includegraphics[width=0.98\linewidth]{fig_gradient_alignment_phase2b_best.pdf}
\caption{\textbf{Phase-2b best-regime gradient fidelity.} Shunting models show substantially better local-vs-backprop directional alignment (higher weighted cosine) and much lower gradient-scale distortion than additive controls on both MNIST and context-gating best configurations.}
\label{fig:gradient_alignment_components}
\end{figure}

\paragraph{Per-soma vs scalar (trained-checkpoint comparison).}
Using the \emph{trained} best Phase-2b checkpoints (\texttt{config\_74} for MNIST, \texttt{config\_153} for context gating), we rerun the same one-batch gradient-fidelity diagnostic with only \texttt{error\_broadcast\_mode} changed between \texttt{per\_soma} and \texttt{scalar}. For shunting cores, directional and scale metrics are nearly identical in this one-step diagnostic (Table~\ref{tab:gradient_mode_compare_ckpt}), suggesting that the large end-task performance gap between broadcast modes is primarily a \emph{training-trajectory} effect rather than a single-step local-gradient direction effect.

\begin{table}[t]
\centering
\small
\begin{tabular}{@{}lcccc@{}}
\toprule
\textbf{Dataset (shunting)} & \textbf{Cosine (per-soma)} & \textbf{Cosine (scalar)} & \textbf{Scale (per-soma)} & \textbf{Scale (scalar)} \\
\midrule
MNIST & 0.0537 & 0.0536 & 1.6756 & 1.6765 \\
Context gating & 0.1140 & 0.1138 & 0.8303 & 0.8314 \\
\bottomrule
\end{tabular}
\caption{\textbf{Broadcast-mode checkpoint comparison on best Phase-2b runs.} ``Scale'' is \(|\log_{10}(\|g_{\mathrm{local}}\|/\|g_{\mathrm{bp}}\|)|\), parameter-count weighted over component groups.}
\label{tab:gradient_mode_compare_ckpt}
\end{table}

\begin{figure}[t]
\centering
\includegraphics[width=0.98\linewidth]{fig_gradient_alignment_phase2b_mode_compare.pdf}
\caption{\textbf{Per-soma vs scalar gradient-fidelity on trained Phase-2b checkpoints.} Aggregate directional and scale metrics for shunting and additive cores on MNIST and context gating. In this one-step diagnostic, mode differences are small relative to core-type differences.}
\label{fig:gradient_alignment_mode_compare}
\end{figure}

\paragraph{Gradient fidelity over training.}
Beyond the single-step diagnostic, we track gradient alignment throughout training by saving checkpoints every 2 epochs and computing the fidelity metrics at each checkpoint (Fig.~\ref{fig:gradient_fidelity_trajectory}). Shunting cores maintain positive cosine similarity ($\approx 0.05$--$0.15$) across all epochs for the 5F rule, while additive cores fluctuate around zero. The per-component analysis (Panel~B) reveals that reactivation parameters achieve near-perfect alignment ($>0.99$) even in additive cores, while excitatory and inhibitory synapses show the largest shunting-vs-additive gap. Scale mismatch (Panel~C) decreases during training for shunting cores, indicating that local and backprop gradient magnitudes converge as the network learns---a form of implicit gradient normalization enabled by the conductance-based architecture.

\begin{figure*}[t]
\centering
\includegraphics[width=0.95\textwidth]{fig_gradient_fidelity_trajectory.pdf}
\caption{\textbf{Gradient fidelity over training.} (A)~Parameter-count-weighted cosine similarity between local and backprop gradients over 50 training epochs. Shunting cores (solid) maintain positive alignment while additive cores (faded) fluctuate near zero. (B)~Per-component cosine similarity at epoch 0 (5F rule): reactivation parameters show near-perfect alignment in both architectures, while synaptic parameters show the largest shunting advantage. (C)~Scale mismatch ($|\log_{10}$ norm ratio$|$) decreases during training for shunting cores, indicating convergence of local and backprop gradient magnitudes. Shaded regions: $\pm 1$ s.d.\ across 3 seeds.}
\label{fig:gradient_fidelity_trajectory}
\end{figure*}

\subsection{Core Empirical Figures}

\begin{figure*}[t]
\centering
\includegraphics[width=0.94\textwidth]{fig_phase1_capacity_best_core.pdf}
\caption{\textbf{Phase 1 capacity ceilings (standard backprop).} Best standard-test accuracy per dataset and core type. This establishes a meaningful performance ceiling before evaluating LocalCA competence and mechanistic effects.}
\label{fig:phase1_capacity}
\end{figure*}

\begin{figure}[t]
\centering
\includegraphics[width=0.95\linewidth]{fig_phase2b_gap_closing_hsic_weight.pdf}
\caption{\textbf{HSIC strength and broadcast mode.} For context gating, moderate HSIC weights ($0.01$--$0.1$) improve local learning under per-soma broadcast, while large weights degrade performance. For MNIST, HSIC has negligible effect. Error bars: $\pm 1$ s.d.\ across 5 seeds.}
\label{fig:phase2b_hsic_weight}
\end{figure}

\begin{figure*}[t]
\centering
\includegraphics[width=0.92\textwidth]{fig_decoder_locality.pdf}
\caption{\textbf{Decoder locality.} On both MNIST and CIFAR-10, local decoder updates match backpropagated decoder updates, while frozen decoder weights collapse performance. This confirms that the decoder can be trained with purely local rules.}
\label{fig:decoder_locality}
\end{figure*}

\begin{figure*}[t]
\centering
\includegraphics[width=0.94\textwidth]{fig_phase3_claimA_shunting_heatmap.pdf}
\caption{\textbf{Shunting advantage across inhibition strength.} Heatmap of shunting minus additive test-accuracy difference over inhibitory synapse count and broadcast modes, showing the regimes where shunting-linked local credit is most beneficial.}
\label{fig:claimA_shunting_heatmap}
\end{figure*}

\begin{figure*}[t]
\centering
\includegraphics[width=0.94\textwidth]{fig_shunting_regime.pdf}
\caption{\textbf{Regime dependence across inhibition levels (robust sweep).} Shunting networks consistently outperform additive controls across inhibitory synapse counts and broadcast modes. This isolates the main performance source as the shunting regime rather than local-learning heuristics alone.}
\label{fig:shunting_regime}
\end{figure*}

\begin{figure*}[t]
\centering
\includegraphics[width=0.94\textwidth]{fig_source_interaction.pdf}
\caption{\textbf{Broadcast-path interaction in source analysis (robust sweep).} Path propagation has interaction-dependent behavior: within per-soma broadcast it leaves accuracy nearly unchanged while increasing information metrics, whereas within scalar broadcast it improves accuracy more with smaller information gains.}
\label{fig:source_interaction}
\end{figure*}

\begin{figure*}[t]
\centering
\includegraphics[width=0.94\textwidth]{fig_phase_information_panel.pdf}
\caption{\textbf{Information panel.} Mutual information $I(E,I;C)$ versus test accuracy for shunting (green) and additive (blue) networks, with (diamonds) and without (circles) path propagation. Dendritic mechanisms can systematically reshape representations even when end-task accuracy is only weakly affected, motivating mechanistic analyses beyond headline accuracy.}
\label{fig:phase_information_panel}
\end{figure*}

\begin{figure*}[t]
\centering
\begin{subfigure}[t]{0.49\textwidth}
\centering
\includegraphics[width=\textwidth]{fig_phase3_claimB_morphology_scaling.pdf}
\caption{\textbf{Morphology scaling.} Accuracy as a function of dendritic branching with and without path propagation/modulators.}
\end{subfigure}
\hfill
\begin{subfigure}[t]{0.49\textwidth}
\centering
\includegraphics[width=\textwidth]{fig_phase3_claimC_error_shaping.pdf}
\caption{\textbf{Error shaping.} Comparison of broadcast modes and decoder update modes on hierarchical/context tasks.}
\end{subfigure}
\caption{\textbf{Phase 3 ablations: morphology and error shaping.} These sweeps identify which LocalCA components and morphology-aware extensions have the largest effect on performance and representation metrics.}
\label{fig:phase3_morph_error}
\end{figure*}

\begin{table}[t]
\centering
\small
\begin{tabular}{@{}lll@{}}
\toprule
\textbf{Robust Claim / Condition} & \textbf{Metric} & \textbf{Value} \\
\midrule
Claim2 (MNIST, decoder local vs backprop vs none) & test accuracy & 0.3790 vs 0.3788 vs 0.1756 \\
Claim2 (CIFAR-10, decoder local vs backprop vs none) & test accuracy & 0.1669 vs 0.1671 vs 0.1000 \\
Claim3 (MNIST, shunting vs additive; avg matched) & test accuracy delta & +0.210 \\
Claim3 (CIFAR-10, shunting vs additive; avg matched) & test accuracy delta & +0.044 \\
Claim4 (per-soma, path true minus false) & test / \(\mathrm{MI}(E,I;C)\) & -0.0030 / +0.0173 \\
\bottomrule
\end{tabular}
\caption{\textbf{Mechanistic ablation results (controlled architecture).} These experiments use a smaller architecture to cleanly isolate individual mechanisms (decoder locality, shunting regime dependence, broadcast/path interaction) rather than maximize absolute performance.}
\label{tab:robust_headline}
\end{table}

\subsection{Primary Metrics and Statistical Protocol}

\begin{itemize}
\item \textbf{Primary endpoint:} test accuracy (mean $\pm$ 95\% CI over seeds).
\item \textbf{Secondary:} validation NLL, convergence speed (epochs to best checkpoint), robustness across MNIST/CIFAR-10.
\item \textbf{Mechanistic:} MI/CMI terms, branch/path statistics, compartment SNR, ablation sensitivity.
\item \textbf{Statistics:} two-way ANOVA (broadcast $\times$ path), interaction contrasts, and paired seed-matched deltas for local-vs-backprop decoder updates.
\end{itemize}


\paragraph{Limitations and future work.}
Several extensions remain: (i)~scaling to deeper architectures and image benchmarks beyond flattened inputs; (ii)~extension to spiking neural networks via eligibility traces (Appendix~\ref{app:eligibility}); (iii)~testing on reconstructed dendritic morphologies from NeuroMorpho; and (iv)~formal convergence analysis under Robbins-Monro conditions.

\section{Related Work}

\paragraph{Dendritic credit assignment.}
Several proposals exploit dendritic morphology for biologically plausible learning. Guerguiev et al.~\cite{guerguiev2017segregated} proposed segregated dendrites where apical compartments carry error signals and basal compartments carry feedforward input, enabling approximate backpropagation in multi-layer networks. Sacramento et al.~\cite{sacramento2018dendritic} showed that cortical microcircuits with dendritic predictions can approximate backprop if apical and basal inputs converge with appropriate timing. Our work differs in two key ways: (i)~we derive exact gradients for conductance-based compartmental models (not abstract multi-compartment neurons), and (ii)~we identify shunting inhibition as a critical enabler of local gradient quality through a quantitative gradient-fidelity analysis.

\paragraph{Feedback alignment and variants.}
Lillicrap et al.~\cite{lillicrap2016random} showed that random fixed feedback weights suffice for learning in deep networks (feedback alignment, FA), and N{\o}kland~\cite{nokland2016dfa} extended this to direct feedback alignment (DFA) where output errors project directly to each layer. Our broadcast error modes (Section~3.1) generalize this idea to the dendritic setting. The key difference is that our local rules also exploit conductance-based local signals (driving force, input resistance) that are unavailable to standard FA/DFA.

\paragraph{Other biologically plausible methods.}
Equilibrium propagation \cite{scellier2017equilibrium} computes exact gradients in energy-based networks by contrasting free and clamped phases. Predictive coding networks \cite{whittington2019theories} perform inference via local prediction errors that converge to backprop gradients at equilibrium. Target propagation methods use local targets rather than error gradients. These approaches solve the weight transport problem through different mechanisms; our contribution is orthogonal, showing that dendritic biophysics provides yet another route to local credit assignment.

\paragraph{Divisive normalization and shunting.}
Shunting inhibition provides divisive normalization of neural responses \cite{carandini2012normalization}, though its effect on firing rates can be subtractive in certain regimes \cite{holt1997shunting}. We show that this divisive interaction has a previously unrecognized benefit for learning: it creates a regime where local gradient approximations are substantially more faithful to exact gradients. Homeostatic mechanisms including synaptic scaling \cite{turrigiano2008homeostatic} and apical/basal plasticity differences \cite{larkum2013apical} motivate our morphology-aware extensions.

\section{Conclusion}

We have shown that compartmental dendritic networks with shunting inhibition create a favorable regime for local credit assignment. Starting from conductance-based dendritic equations, we derived exact backpropagation gradients for dendritic trees and constructed a principled hierarchy of local approximations (3F/4F/5F) that use only synapse-local quantities plus a broadcast error signal. Our central empirical finding is that \emph{shunting inhibition is the key architectural enabler}: it provides divisive normalization that dramatically improves the directional alignment and scale fidelity of local gradients relative to exact backpropagation. The best local rule (5F with per-soma broadcast) closes much of the gap to backpropagation on standard benchmarks, and the gradient-fidelity diagnostic provides a new tool for understanding \emph{why} certain architectures support local learning better than others.

\appendix

\section{Morphology-Aware Extensions (Details)}
\label{app:morphology}

This appendix provides the full definitions for the morphology-aware extensions summarized in the main text.

\paragraph{Branch-specific depth modulation.}
Let $d_j$ be the graph distance from the soma to branch $j$. Define per-branch morphology factor $\rho_j = \rho_{\mathrm{base}} / (d_j + \alpha)$, where $\alpha > 0$ prevents singularity. This mirrors cable attenuation: distal synapses receive smaller plasticity updates, with $\|\Delta g_j^{\mathrm{syn}}\| \propto 1/(d_j + \alpha)$.

\paragraph{Dendritic normalization.}
Normalize dendritic updates by total branch conductance: $\Delta g_j^{\mathrm{den}} \leftarrow \Delta g_j^{\mathrm{den}} / (\sum_{k} g_k^{\mathrm{den}} + \varepsilon)$. This reduces update variance when total conductance $G_n$ is large, analogous to homeostatic synaptic scaling \cite{turrigiano2008homeostatic}.

\paragraph{Apical vs basal branch differentiation.}
Assign each branch a type flag $t_j \in \{0, 1\}$ (basal, apical) with type-specific scales $s_j = s_{\mathrm{basal}} + t_j (s_{\mathrm{apical}} - s_{\mathrm{basal}})$. Setting $s_{\mathrm{apical}} > s_{\mathrm{basal}}$ amplifies top-down learning, consistent with distinct plasticity rules in apical vs.\ basal dendrites of pyramidal neurons \cite{larkum2013apical}.

\section{HSIC Auxiliary Objectives (Details)}
\label{app:hsic}

For layer activations $\mathbf{Z} \in \mathbb{R}^{B \times d_n}$ with kernel matrix $\mathbf{K}_Z$ and centering matrix $\mathbf{H} = \mathbf{I} - \frac{1}{B}\mathbf{1}\mathbf{1}^\top$, the HSIC losses are:
\begin{align}
\mathcal{L}_{\mathrm{HSIC}}^{\mathrm{self}} &= \frac{1}{B^2} \tr(\mathbf{K}_Z \mathbf{H} \mathbf{K}_Z \mathbf{H}) & \text{(self-decorrelation)} \\
\mathcal{L}_{\mathrm{HSIC}}^{\mathrm{target}} &= -\frac{1}{B^2} \tr(\mathbf{K}_Z \mathbf{H} \mathbf{K}_Y \mathbf{H}) & \text{(target-correlation)}
\end{align}
For linear kernel $\mathbf{K}_Z = \mathbf{Z}\mathbf{Z}^\top$, the gradients are $\partial \mathcal{L}^{\mathrm{self}} / \partial \mathbf{Z} = \frac{4}{B^2} \mathbf{H} \mathbf{K}_Z \mathbf{H} \mathbf{Z}$ and $\partial \mathcal{L}^{\mathrm{target}} / \partial \mathbf{Z} = -\frac{4}{B^2} \mathbf{H} \mathbf{K}_Y \mathbf{H} \mathbf{Z}$. These are added to synaptic eligibility traces via the chain rule.

\section{Online Variant with Eligibility Traces}
\label{app:eligibility}

Define continuous-time eligibilities per synapse:
$\tau_e \dot{e}_{j}^{\mathrm{syn}}(t) = -e_{j}^{\mathrm{syn}}(t) + x_j(t)\, (E_j - V_n(t))\, R_n^{\mathrm{tot}}(t)$,
and likewise for dendritic connections. With modulatory signal $m_n(t)$:
$\Delta g_j^{\mathrm{syn}} \propto \int e_{j}^{\mathrm{syn}}(t)\, m_n(t)\, \mathrm{d}t$,
which instantiates three-factor learning in continuous time \cite{fremaux2016three,bellec2020eprop}.

\begin{thebibliography}{99}

\bibitem{koch1999biophysics}
Koch, C. (1999).
\emph{Biophysics of Computation: Information Processing in Single Neurons}.
Oxford University Press.

\bibitem{dayan2001theoretical}
Dayan, P., \& Abbott, L. F. (2001).
\emph{Theoretical Neuroscience}.
MIT Press.

\bibitem{carandini2012normalization}
Carandini, M., \& Heeger, D. J. (2012).
Normalization as a canonical neural computation.
\emph{Nature Reviews Neuroscience}, 13(1), 51--62.

\bibitem{holt1997shunting}
Holt, G. R., \& Koch, C. (1997).
Shunting inhibition does not have a divisive effect on firing rates.
\emph{Neural Computation}, 9(5), 1001--1013.

\bibitem{lillicrap2016random}
Lillicrap, T. P., Cownden, D., Tweed, D. B., \& Akerman, C. J. (2016).
Random synaptic feedback weights support error backpropagation for deep learning.
\emph{Nature Communications}, 7, 13276.

\bibitem{nokland2016dfa}
N{\o}kland, A. (2016).
Direct feedback alignment provides learning in deep neural networks.
\emph{Advances in Neural Information Processing Systems}, 29.

\bibitem{guerguiev2017segregated}
Guerguiev, J., Lillicrap, T. P., \& Richards, B. A. (2017).
Towards deep learning with segregated dendrites.
\emph{eLife}, 6, e22901.

\bibitem{sacramento2018dendritic}
Sacramento, J., Costa, R. P., Bengio, Y., \& Senn, W. (2018).
Dendritic cortical microcircuits approximate the backpropagation algorithm.
\emph{Advances in Neural Information Processing Systems}, 31.

\bibitem{whittington2019theories}
Whittington, J. C., \& Bogacz, R. (2019).
Theories of error back-propagation in the brain.
\emph{Trends in Cognitive Sciences}, 23(3), 235--250.

\bibitem{scellier2017equilibrium}
Scellier, B., \& Bengio, Y. (2017).
Equilibrium propagation: Bridging the gap between energy-based models and backpropagation.
\emph{Frontiers in Computational Neuroscience}, 11, 24.

\bibitem{gretton2005hsic}
Gretton, A., Bousquet, O., Smola, A., \& Sch{\"o}lkopf, B. (2005).
Measuring statistical dependence with Hilbert-Schmidt norms.
\emph{International Conference on Algorithmic Learning Theory}, 63--77.

\bibitem{gretton2007hsic}
Gretton, A., Fukumizu, K., Teo, C. H., Song, L., Sch{\"o}lkopf, B., \& Smola, A. J. (2007).
A kernel statistical test of independence.
\emph{Advances in Neural Information Processing Systems}, 20.

\bibitem{fremaux2016three}
Fr{\'e}maux, N., \& Gerstner, W. (2016).
Neuromodulated spike-timing-dependent plasticity, and theory of three-factor learning rules.
\emph{Frontiers in Neural Circuits}, 9, 85.

\bibitem{bellec2020eprop}
Bellec, G., Scherr, F., Subramoney, A., Hajek, E., Salaj, D., Legenstein, R., \& Maass, W. (2020).
A solution to the learning dilemma for recurrent networks of spiking neurons.
\emph{Nature Communications}, 11, 3625.

\bibitem{larkum2013apical}
Larkum, M. (2013).
A cellular mechanism for cortical associations: an organizing principle for the cerebral cortex.
\emph{Trends in Neurosciences}, 36(3), 141--151.

\bibitem{welford1962note}
Welford, B. P. (1962).
Note on a method for calculating corrected sums of squares and products.
\emph{Technometrics}, 4(3), 419--420.

\bibitem{turrigiano2008homeostatic}
Turrigiano, G. G. (2008).
The self-tuning neuron: synaptic scaling of excitatory synapses.
\emph{Cell}, 135(3), 422--435.

\end{thebibliography}

\end{document}
